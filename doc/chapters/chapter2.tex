\chapter{First-Order Logic}
\label{ch:First-Order Logic}

First Order Logic is a formal system that deals with logical statements 
that are more complex than propositional logic.

It is composed of a set of symbols that are used to represent logical statements,
and a set of rules that define how these symbols can be combined to form logical statements.

It introduce the concept of \textit{variables}, \textit{functions} and \textit{predicates},
which allows us to reason about objects and relations between objects.

\section{Syntax in First-Order Logic}
\label{sec:Syntax in First-Order Logic}

\begin{definition}{Signature}
    A \textbf{signature} is a tuple $\sig = (\consSet,\funcSet,\predSet)$ where:
    \begin{itemize}
        \item $\consSet$ is a set of constant symbols.
        \item $\funcSet$ is a set of function symbols.
        \item $\predSet$ is a set of predicate symbols.
    \end{itemize}
\end{definition}

The \textbf{constant} symbols denote individual elements (e.g. $0$, $1$, $a$, $b$, \dots).

The \textbf{function} symbols denote functions that take a number of arguments 
and return a value (e.g. $+$, $\times$, $f$, $g$, \dots).

The \textbf{predicate} symbols denote relations that take a number of arguments
and return a truth value (e.g. $=$, $<$, $P$, $Q$, \dots).

The elements of $\consSet$, $\funcSet$ and $\predSet$ are called \textit{symbols}.

The arity of a function or predicate symbol is the number of arguments it takes.

The difference between a function and a predicate lies in their 
connection, infact a function returns a value while a predicate returns a truth value.

\begin{example}{Predicate vs Function}
    $f(x) = f(y)$ these functions are connected by the equality predicate $=$, 
    while these predicates $P(x) \iff P(y)$ are connected by logical equivalence $\iff$
\end{example}

\begin{remark}{Functions of Predicate}
    A predicate $P$ can be seen as a function $f_P$ that returns a truth value.
    So introducing a constant symbol
    $\circ$ witch denotes ``truth'' than we can define $f_P$ as:
    \begin{equation*}
        f_P(x_1,\dots,x_n) = \circ \quad \text{if } P(x_1,\dots,x_n) \text{ is true}
    \end{equation*} 
    where $\circ$ is added to the signature. 
    $\sig' = \sig \cup \{\circ\}$  
\end{remark}

In FOL, there are logical connectives that 
are used to combine logical statements:
$\neg$ (negation), $\land$ (conjunction), $\lor$ (disjunction), 
$\implies$ (implication), $\iff$ (equivalence).

Using a signature $\sig$ and 
a set of variables $\varSet$, we can define the syntax of FOL.

\begin{definition}{Term}
    Let $\sig$ be a signature and $\varSet$ a set of variables.
    A \textbf{term} is defined as follows:
    \begin{itemize}
        \item Every constant symbol $c \in \consSet$ is a term.
        \item Every variable $x \in \varSet$ is a term.
        \item Every $n$-ary function symbol $f \in \funcSet$ with $t_1,\dots,t_n$ are terms, 
        then $f(t_1,\dots,t_n)$ is a term.
    \end{itemize}
\end{definition}

\begin{definition}{Atom}
    Let $\sig$ be a signature and $\varSet$ a set of variables.
    An \textbf{atom} is defined as follows:
    \begin{itemize}
        \item Every $n$-ary predicate symbol $P \in \predSet$ with $t_1,\dots,t_n$ are terms, 
        then $P(t_1,\dots,t_n)$ is an atom.
    \end{itemize}    
\end{definition}

\begin{definition}{Literal}
    A \textbf{literal} is an atom or the negation of an atom.
\end{definition}

\begin{definition}{Formula}
    Let $\sig$ be a signature and $\varSet$ a set of variables.
    A \textbf{formula} is defined as follows:
    \begin{itemize}
        \item Every atom is a formula.
        \item If $F$ and $G$ are formulas, then 
        $\neg F$, $F \land G$, $F \lor G$, $F \implies G$, $F \iff G$ are formulas.
        \item If $F$ is a formula and $x \in \varSet$ is a variable, 
        then $\forall x.F$ and $\exists x.F$ are formulas.
    \end{itemize}
\end{definition}

From a formula we can destinguish the \textit{free variables} and the \textit{bound variables}.
The free variables are the variables that are not bounded by a quantifier,
while the bound variables are.

\begin{example}{Free and Bound Variables}
    Give the formula $H$:
    \begin{equation*}
        f(x) = b \land \forall y. f(y) = b
        \implies f(f(y)) = b
    \end{equation*}
    We can define the free variables as $FV(H) = \{x\}$ and the bound variables as 
    $BV(H) = \{y\}$. While $b$ is a constant symbol.
\end{example}

\begin{remark}{Renaming Variables}
    Remark that a variable cannot be both free and bound at the same time.
    There is a issue with the renaming of variables:
    free variables cannot be renamed, while bound variables can, 
    but the it cannot be renamed to a variable that is already in the formula.
\end{remark}

[IMMAGINE VALIDITY PROBLEM IN FOL ][NOTE IN SIGNATURE 3 OTTOBRE]

\section{Semantics in First-Order Logic}
\label{sec:Semantics in First-Order Logic}

\begin{definition}{Interpretation}
    Let $\sig$ be a signature.
    An \textbf{interpretation} $\I$ of $\sig$ is a tuple $\I = (\dom,\inteFunc)$ where:
    \begin{itemize}
        \item $\dom$ is a non-empty set called the \textit{domain} of $\I$.
        \item $\inteFunc$ is a function that assigns to each symbol in $\sig$ 
        an element of $\dom$.
    \end{itemize}
    The function $\inteFunc$ is defined as follows:
    \begin{itemize}
        \item $\forall c \in \consSet$, $\inteFunc(c) \in \dom$.
        \item $\forall f \in \funcSet$, 
        $\inteFunc(f) : \dom^n \to \dom$.
        \item $\forall P \in \predSet$, 
        $\inteFunc(P) : \dom^n$.
    \end{itemize}
    Where $n$ is the arity of the function or predicate symbol.
\end{definition}

So an interpretation assigns a meaning to the symbols in the signature, 
there can be multiple interpretations for the same signature.

\begin{example}{Interpretation of Integers}
    Let $\sig = (\{a,b\},\{f\},\{R\})$ be a signature.
    An interpretation $\I$ of $\sig$ can be defined as follows:
    \begin{itemize}
        \item $\dom = \mathbb{Z}$.
        \item $\inteFunc(a) = -3$.
        \item $\inteFunc(b) = 3$.
        \item $\inteFunc(f) = + : \mathbb{Z}^2 \to \mathbb{Z}$ is the addition function.
        \item $\inteFunc(R) = \geq : \mathbb{Z}^2$ is the greater than or equal to predicate.
    \end{itemize}
\end{example}

\begin{example}{Interpretation of Color}
    Let $\sig = (\{a,b,c\},\emptyset,\{R\})$ be a signature.
    An interpretation $\I$ of $\sig$ can be defined as follows:
    \begin{itemize}
        \item $\dom = \{red,green,blue\}$.
        \item $\inteFunc(a) = red$.
        \item $\inteFunc(b) = blue$.
        \item $\inteFunc(c) = green$.
        \item $\inteFunc(R) = \{(red,blue),(red,red)\}$.
    \end{itemize}
\end{example}

A term $t$ is evaluated in an interpretation $\I$ as follows:
\begin{equation*}
    [t]_\I = 
    \begin{cases*}
        \inteFunc(c) & if $t = c \in \consSet$ is a constant symbol \\
        \inteFunc(f)([t_1]_\I,\dots,[t_n]_\I) & if $t = f(t_1,\dots,t_n) \in \funcSet$ 
        is a function symbol \\
        \assignFunc(x) & if $t = x \in \varSet$ is a variable
    \end{cases*}
\end{equation*}
Where $n$ is the arity of the function symbol $f$ and 
$\assignFunc$ is an assignment function that assigns a value to a variable.

\section{Satifaction and Validity in First-Order Logic}
\label{sec:Satifaction and Validity in First-Order Logic}

A formula $F$ is evaluated in an interpretation $\I$ and 
it is said to be \textit{satisfied} in $\I$ if it evaluates to true,
noted as $\I \vDash  F$, otherwise it is said to be \textit{unsatisfied} in $\I$,
noted as $\I \nvDash F$.

The first-order formulas follows the same definition of satisfaction as the 
propositional formulas:

\begin{itemize}
    \item \textbf{Satisfiable Formula} (\ref{def:Satisfiable Formula}).
    \item \textbf{Valid Formula} (\ref{def:Valid Formula}).
    \item \textbf{Unsatisfiable Formula} (\ref{rem:Unsatisfiable and Invalid Formulas}).
    \item \textbf{Invalid Formulas} (\ref{rem:Unsatisfiable and Invalid Formulas}).
    \item \textbf{Implication of Validity} (\ref{rem:Implication of Validity}).
\end{itemize}

But the satisfiable relation for first-order formulas are defined as follows:
Let $F,G,H$ be formulas and $\I$ be an interpretation, than:
\begin{itemize}
    \item $\I \vDash \neg F$ if $\I \nvDash F$.
    \item $\I \vDash F \land G$ if $\I \vDash F \land \I \vDash G$.
    \item $\I \vDash F \lor G$ if $\I \vDash F \lor \I \vDash G$.
    \item $\I \vDash F \implies G$ if $\I \nvDash F \implies \I \vDash G$.
    \item $\I \vDash F \iff G$ if $\I \vDash F \iff \I \vDash G$.
    \item $\I \vDash \forall x.F$ if 
    $\forall d \in \dom : \I \vDash_{\assignFunc[x\rightarrow d]} G$  
    \item $\I \vDash \exists x.F$ if
    $\exists d \in \dom : \I \vDash_{\assignFunc[x\rightarrow d]} G$
\end{itemize}

Where $\I \vDash_{\assignFunc[x\rightarrow d]} G$ means that the formula $G$ is satisfied in $\I$
with the assignment function $\assignFunc$ that assigns the value $d$ to the variable $x$.
\begin{equation*}
    \assignFunc[x\rightarrow d](y) = 
    \begin{cases*}
        d & if $y = x$ \\
        \assignFunc(y) & otherwise
    \end{cases*}
    \forall y \in \varSet
\end{equation*}

\section{Theories in First-Order Logic}
\label{sec:Theories in First-Order Logic}

A \textbf{theory} formalize structures in a specific domain of interest, 
and help us reason about the properties of these structures.
It really useful in verification.

Will be introduced some definitions that concerns theories in FOL.

\begin{definition}{Theory}
    A \textbf{theory} $\theory$ is definted as a tuple $\theory = (\sig,\axioms)$ where:
    \begin{itemize}
        \item $\sig$ is a signature.
        \item $\axioms$ is a set of formulas called \textit{axioms},
        with only elements of the signature.
    \end{itemize}
\end{definition}

\begin{definition}{Sigma-Forumla}
    A formula $F$ is a \textbf{$\sig$-formula} if it contains symbols in
    the signature $\sig$,
    as well as the logical connectives, quantifiers and variables.
\end{definition}

\begin{definition}{Theory-Interpretation}
    If $\I$ is an interpretation of $\sig$, then $\I$ is a \textbf{$\theory$-interpretation}
    of a theory $\theory = (\sig,\axioms)$ if $\I \vDash \axioms$.
\end{definition}

\begin{definition}{Theory-Satisfiable Formula}
    Let $\theory = (\sig,\axioms)$ be a theory. And $F$ be a $\sig$-formula.
    If $\exists \I$ interpretation of $\sig$:
    \begin{equation*}
        \I \vDash \axioms \land \I \vDash F
    \end{equation*}
    Which means that $F$ is satisfied in $\I$ and $\I$ is a $\theory$-interpretation.

    So $F$ is \textbf{$\theory$-satisfiable} in the theory $\theory$.
\end{definition}

\begin{definition}{Theory-Vaild Formula}
    Let $\theory = (\sig,\axioms)$ be a theory. And $F$ be a $\sig$-formula.
    If $\forall \I$ interpretation of $\sig$:
    \begin{equation*}
        \I \vDash \axioms \implies \I \vDash F
    \end{equation*}

    Which means $F$ is valid ($\vDash F$) in the theory $\theory$ if 
    every $\theory$-interpretation satisfies $F$.

    Then $F$ is a \textbf{$\theory$-valid formula}, also 
    noted as $\theory \vDash F$. 
\end{definition}

\begin{definition}{Theory Fragment}
    A \textbf{theory fragment} is a theory that deals only with a subset 
    of formulas of the original theory.
\end{definition}

\begin{definition}{Quantifier-Free Fragment}
    The \textbf{quantifier-free fragment} of a theory $\theory$ is the theory 
    that contains only the formulas that do not containg quantifier: 
    $\forall$ and $\exists$.
    Which means that the variables in the formulas are free.
\end{definition}

\subsection{Theory of Equality}
\label{subsec:Theory of Equality}

The theory of equality is a theory that is centered around 
the \textbf{equality predicate} $\eql$ and the equivalence
axioms.

\begin{definition}{Equivalence Axioms}
    Let $\varSet$ be a set of variables and $\eql$ be a predicate symbol.
    Let $\axioms$ be a set of formulas called \textbf{equivalence axioms} if 
    it contains the following formulas:
    \begin{itemize}
        \item \textbf{Reflexivity}: $\forall x \in \varSet. \quad x \eql x$.
        \item \textbf{Symmetry}: $\forall x,y \in \varSet. \quad x \eql y \implies y \eql x$.
        \item \textbf{Transitivity}: 
        $\forall x,y,z \in \varSet. \quad x \eql y \land y \eql z \implies x \eql z$.
    \end{itemize}
\end{definition}

\begin{definition}{Congruence Axioms}
    Let $\varSet$ be a set of variables and $\eql$ be a predicate symbol.
    Let $\funcSet$ be a set of function symbols and $\predSet$ be a set of predicate symbols.
    Let $\axioms$ be a set of formulas called \textbf{congruence axioms} if
    it contains the following formulas:
    \begin{itemize}
        \item \textbf{Function Congruence}: 
        $\forall$ function symbol $f \in \funcSet$ with arity $n$ and
        $\forall x_1,\dots,x_n,y_1,\dots,y_n \in \varSet$:
        \begin{equation*}
            x_1 \eql y_1 \land \dots \land x_n \eql y_n 
            \implies f(x_1,\dots,x_n) \eql f(y_1,\dots,y_n)
        \end{equation*}
        \item \textbf{Predicate Congruence}:
        $\forall$ predicate symbol $R \in \predSet$ with arity $n$ and
        $\forall x_1,\dots,x_n,y_1,\dots,y_n \in \varSet$:
        \begin{equation*}
            x_1 \eql y_1 \land \dots \land x_n \eql y_n 
            \implies R(x_1,\dots,x_n) \iff R(y_1,\dots,y_n)
        \end{equation*}
    \end{itemize}
    Which states that if the arguments of a function or predicate are equal, 
    then the result of the function or the truth value of the predicate is equal.
\end{definition}

\begin{definition}{Theory of Equality}
    Let $\eqSig = (\consSet,\funcSet,\predSet \cup \{\eql\})$ be a signature.
    Let $\eqAxi$ be a set of formulas called \textbf{equality axioms} 
    if it contains the Equivalence Axioms and the Congruence Axioms.
    The it can be defined the \textbf{theory of equality} as:
    \begin{equation*}
        \eqTheory = (\eqSig,\eqAxi)
    \end{equation*}
\end{definition}

\begin{notation}{Disequaliti in Theory of Equality}
    It is possible to define the \textbf{disequality predicate} $\deql$ as:
    \begin{equation*}
        x \deql y \iff \neg (x \eql y)
    \end{equation*}
    Where $x$ and $y$ are variables in $\varSet$.
    It also possibleto ridefine the equality signature as:
    \begin{equation*}
        \eqSig = (\consSet,\funcSet,\predSet \cup \{\eql,\deql\})
    \end{equation*}
\end{notation}

\begin{remark}{Theory of Equality}
    Since the equality axioms contanin the equivalence axioms and 
    the congruence axioms, then the equality predicate $\eql$ is
    a \textbf{congruence relation}.
\end{remark}

\begin{example}{Satisfiability in Theory of Equality}
    The formula $a \eql b \land f(a) \eql f(b)$ is
    satisfiable in the theory of equality ($\eqTheory$-satisfiable).
    While the formula $a \eql b \land R(a) \iff \neg R(b)$ is
    unsatisfiable in the theory of equality
    ($\eqTheory$-unsatisfiable).
\end{example}

\subsection{Logical Consequence}
\label{subsec:Logical Consequence}

\textbf{Logical Consequence} is the relation between a set of formulas (assumptions) and 
a formula (conjecture), where the conjecture is true if the assumptions are true.

Let $H$ be a set of formulas, called ``\textit{assumption}'',
and $\varphi$ be a formula, called ``\textit{conjecture}'':

We have that $H \vDash \varphi$ or equivalently $\vDash H \implies \varphi$,
which means that $\varphi$ is \textbf{logica consequence} of $H$, then:
\begin{equation*}
    \forall \I \text{ interpretation }: \I \vDash H 
    \iff \I \vDash \varphi \iff H \cup \{\neg \varphi\} \text{ is unsatisfiable}
\end{equation*}

The last coimplication derives from the fact that $\neg \varphi$ is the negation of 
the conjecture, so if we find an interpretation that satisfies all the formulas in $H$ 
then it must satisfy the conjecture, making the satisfaction of 
$\neg \varphi$ impossible.

We need a way to determine if a formula is a logical consequence of a set of formulas:
we can build a decision procedure that checks if the set of formulas is unsatisfiable,
in particular the procedure checks if $H \cup \{\neg \varphi\} \vDash \bot$.

[IMMGINE CON PROCEDURA iN 8 OTTOBRE]

\begin{notation}{Nested Function}
    The notation $f^{(n)}(x)$ denotes the 
    $n$-th iteration of the function $f$ on the argument $x$:
    \begin{equation*}
        f^{(n)}(x) = \underbrace{f(f(\dots f(x) \dots))}_{n}
    \end{equation*}
\end{notation}

\begin{example}{Human Reasoning for Satisfiability}
    Let the formula:
    \begin{equation*}
        F = \underbrace{f^{(3)}(x) \eql x}_{\text{EQ1}} \land
        \underbrace{f^{(5)}(x) \eql x}_{\text{EQ2}} \land 
        \underbrace{f(x) \deql x}_{\text{EQ3}}
    \end{equation*}
    Using human reasoning we can see that:
    \begin{itemize}
        \item EQ1 + EQ2 $\implies$ $f^{(2)}(x) \eql x$ (EQ4).
        \item EQ1 + EQ4 $\implies$ $f(x) \eql x$ (EQ5).
        \item EQ3 + EQ5 $\implies$ $\bot$.
    \end{itemize}
    So the formula $F$ is unsatisfiable.
\end{example}

The human reasoning is not efficient for large formulas, and 
for machines, we need a decision procedure that can determine
the satisfiability with an algorithm that can be executed by a computer.

