\chapter{Syntax and Semantics of First-Order Logic}
\label{ch:Syntax and Semantics of First-Order Logic}

Automated Reasoning is the study of algorithms and systems that allow computers to reason about 
logical statements. 

In this chapter, we will introduce the syntax and semantics of First-Order Logic (FOL), 
which is the most widely used logic in the field of Automated Reasoning. 

We will also introduce the concept of a \textit{model} and 
the notion of \textit{validity} of a logical statement.

Automated Reasoning is achieved by using symbol reasoning, which is the manipulation of symbols
according to the rules of logic.

\section{Signatures and Terms in First-Order Logic}
\label{sec:Signatures and Terms in First-Order Logic}

\begin{definition}{Signature}
    A \textbf{signature} is a tuple $\sig = (\consSet,\funcSet,\predSet)$ where:
    \begin{itemize}
        \item $\consSet$ is a set of constant symbols.
        \item $\funcSet$ is a set of function symbols.
        \item $\predSet$ is a set of predicate symbols.
    \end{itemize}
\end{definition}

The \textbf{constant} symbols denote individual elements (e.g. $0$, $1$, $a$, $b$, \dots).

The \textbf{function} symbols denote functions that take a number of arguments 
and return a value (e.g. $+$, $\times$, $f$, $g$, \dots).

The \textbf{predicate} symbols denote relations that take a number of arguments
and return a truth value (e.g. $=$, $<$, $P$, $Q$, \dots).

The elements of $\consSet$, $\funcSet$ and $\predSet$ are called \textit{symbols}.

The arity of a function or predicate symbol is the number of arguments it takes.

The difference between a function and a predicate lies in their 
connection, infact a function returns a value while a predicate returns a truth value.

\begin{example}{Predicate vs Function}
    $f(x) = f(y)$ these functions are connected by the equality predicate $=$, 
    while these predicates $P(x) \iff P(y)$ are connected by logical equivalence $\iff$
\end{example}

\begin{remark}{Functions of Predicate}
    A predicate $P$ can be seen as a function $f_P$ that returns a truth value.
    So introducing a constant symbol
    $\circ$ witch denotes ``truth'' than we can define $f_P$ as:
    \begin{equation*}
        f_P(x_1,\dots,x_n) = \circ \quad \text{if } P(x_1,\dots,x_n) \text{ is true}
    \end{equation*} 
    where $\circ$ is added to the signature. 
    $\sig' = \sig \cup \{\circ\}$  
\end{remark}

In FOL, there are logical connectives that 
are used to combine logical statements:
$\neg$ (negation), $\land$ (conjunction), $\lor$ (disjunction), 
$\implies$ (implication), $\iff$ (equivalence).

Using a signature $\sig$ and 
a set of variables $\varSet$, we can define the syntax of FOL.

\begin{definition}{Term}
    Let $\sig$ be a signature and $\varSet$ a set of variables.
    A \textbf{term} is defined as follows:
    \begin{itemize}
        \item Every constant symbol $c \in \consSet$ is a term.
        \item Every variable $x \in \varSet$ is a term.
        \item Every $n$-ary function symbol $f \in \funcSet$ with $t_1,\dots,t_n$ are terms, 
        then $f(t_1,\dots,t_n)$ is a term.
    \end{itemize}
\end{definition}

\begin{definition}{Atom}
    Let $\sig$ be a signature and $\varSet$ a set of variables.
    An \textbf{atom} is defined as follows:
    \begin{itemize}
        \item Every $n$-ary predicate symbol $P \in \predSet$ with $t_1,\dots,t_n$ are terms, 
        then $P(t_1,\dots,t_n)$ is an atom.
    \end{itemize}    
\end{definition}

\begin{definition}{Literal}
    A \textbf{literal} is an atom or the negation of an atom.
\end{definition}

\begin{definition}{Formula}
    Let $\sig$ be a signature and $\varSet$ a set of variables.
    A \textbf{formula} is defined as follows:
    \begin{itemize}
        \item Every atom is a formula.
        \item If $F$ and $G$ are formulas, then 
        $\neg F$, $F \land G$, $F \lor G$, $F \implies G$, $F \iff G$ are formulas.
        \item If $F$ is a formula and $x \in \varSet$ is a variable, 
        then $\forall x.F$ and $\exists x.F$ are formulas.
    \end{itemize}
\end{definition}

From a formula we can destinguish the \textit{free variables} and the \textit{bound variables}.
The free variables are the variables that are not bounded by a quantifier,
while the bound variables are.

\begin{example}{Free and Bound Variables}
    Give the formula $H$:
    \begin{equation*}
        f(x) = b \land \forall y. f(y) = b
        \implies f(f(y)) = b
    \end{equation*}
    We can define the free variables as $FV(H) = \{x\}$ and the bound variables as 
    $BV(H) = \{y\}$. While $b$ is a constant symbol.
\end{example}

\begin{remark}{Renaming Variables}
    Remark that a variable cannot be both free and bound at the same time.
    There is a issue with the renaming of variables:
    free variables cannot be renamed, while bound variables can, 
    but the it cannot be renamed to a variable that is already in the formula.
\end{remark}

[IMMAGINE VALIDITY PROBLEM IN FOL ][NOTE IN SIGNATURE 3 OTTOBRE]

\section{Interpretation in First-Order Logic}

\begin{definition}{Interpretation}
    Let $\sig$ be a signature.
    An \textbf{interpretation} $\I$ of $\sig$ is a tuple $\I = (\dom,\inteFunc)$ where:
    \begin{itemize}
        \item $\dom$ is a non-empty set called the \textit{domain} of $\I$.
        \item $\inteFunc$ is a function that assigns to each symbol in $\sig$ 
        an element of $\dom$.
    \end{itemize}
    The function $\inteFunc$ is defined as follows:
    \begin{itemize}
        \item $\forall c \in \consSet$, $\inteFunc(c) \in \dom$.
        \item $\forall f \in \funcSet$, 
        $\inteFunc(f) : \dom^n \to \dom$.
        \item $\forall P \in \predSet$, 
        $\inteFunc(P) : \dom^n$.
    \end{itemize}
    Where $n$ is the arity of the function or predicate symbol.
\end{definition}

So an interpretation assigns a meaning to the symbols in the signature, 
there can be multiple interpretations for the same signature.

\begin{example}{Interpretation of Integers}
    Let $\sig = (\{a,b\},\{f\},\{R\})$ be a signature.
    An interpretation $\I$ of $\sig$ can be defined as follows:
    \begin{itemize}
        \item $\dom = \mathbb{Z}$.
        \item $\inteFunc(a) = -3$.
        \item $\inteFunc(b) = 3$.
        \item $\inteFunc(f) = + : \mathbb{Z}^2 \to \mathbb{Z}$ is the addition function.
        \item $\inteFunc(R) = \geq : \mathbb{Z}^2$ is the greater than or equal to predicate.
    \end{itemize}
\end{example}

\begin{example}{Interpretation of Color}
    Let $\sig = (\{a,b,c\},\emptyset,\{R\})$ be a signature.
    An interpretation $\I$ of $\sig$ can be defined as follows:
    \begin{itemize}
        \item $\dom = \{red,green,blue\}$.
        \item $\inteFunc(a) = red$.
        \item $\inteFunc(b) = blue$.
        \item $\inteFunc(c) = green$.
        \item $\inteFunc(R) = \{(red,blue),(red,red)\}$.
    \end{itemize}
\end{example}

A term $t$ is evaluated in an interpretation $\I$ as follows:
\begin{equation*}
    [t]_\I = 
    \begin{cases*}
        \inteFunc(c) & if $t = c \in \consSet$ is a constant symbol \\
        \inteFunc(f)([t_1]_\I,\dots,[t_n]_\I) & if $t = f(t_1,\dots,t_n) \in \funcSet$ 
        is a function symbol \\
        \assignFunc(x) & if $t = x \in \varSet$ is a variable
    \end{cases*}
\end{equation*}
Where $n$ is the arity of the function symbol $f$ and 
$\assignFunc$ is an assignment function that assigns a value to a variable.

\section{Satifaction and Validity in First-Order Logic}
\label{sec:Satifaction and Validity in First-Order Logic}

A formula $F$ is evaluated in an interpretation $\I$ and 
it is said to be \textit{satisfied} in $\I$ if it evaluates to true,
noted as $\I \vDash  F$, otherwise it is said to be \textit{unsatisfied} in $\I$,
noted as $\I \nvDash F$.

\begin{definition}{Satisfiable Formula}
    A formula $F$ is \textbf{satisfiable} if $\exists \I$ 
    interpretation such that $\I \vDash F$.
\end{definition}

\begin{definition}{Valid Formula}
    A formula $F$ is \textbf{valid} if $\forall \I$ 
    interpretation such that $\I \vDash F$.
\end{definition}

\begin{remark}{Unsatisfiable and Invalid Formulas}
    A formula $F$ is \textbf{unsatisfiable} if $\forall \I$ interpretation 
    such that $\I \nvDash F$.
    A formula $F$ is \textbf{invalid} if $\exists \I$ interpretation 
    such that $\I \nvDash F$.
\end{remark}

\begin{remark}{Implication of Validity}
    Let $F$ be a formula, than we can observe that:
    \maketable{
        \hline
        $F$ & & $\neg F$ \\
        \hline
        Satisfiable & $\implies$ & Invalid \\
        Valid & $\implies$ & Unsatisfiable \\
        Invalid & $\implies$ & Satisfiable \\
        Unsatisfiable & $\implies$ & Valid \\
        \hline
    }{}{|c|c|c|}
\end{remark}

Also the satisfiable relation can be defined as follows:
Let $F,G,H$ be formulas and $\I$ be an interpretation, than:
\begin{itemize}
    \item $\I \vDash \neg F$ if $\I \nvDash F$.
    \item $\I \vDash F \land G$ if $\I \vDash F \land \I \vDash G$.
    \item $\I \vDash F \lor G$ if $\I \vDash F \lor \I \vDash G$.
    \item $\I \vDash F \implies G$ if $\I \nvDash F \implies \I \vDash G$.
    \item $\I \vDash F \iff G$ if $\I \vDash F \iff \I \vDash G$.
    \item $\I \vDash \forall x.F$ if 
    $\forall d \in \dom : \I \vDash_{\assignFunc[x\rightarrow d]} G$  
    \item $\I \vDash \exists x.F$ if
    $\exists d \in \dom : \I \vDash_{\assignFunc[x\rightarrow d]} G$
\end{itemize}

Where $\I \vDash_{\assignFunc[x\rightarrow d]} G$ means that the formula $G$ is satisfied in $\I$
with the assignment function $\assignFunc$ that assigns the value $d$ to the variable $x$.
\begin{equation*}
    \assignFunc[x\rightarrow d](y) = 
    \begin{cases*}
        d & if $y = x$ \\
        \assignFunc(y) & otherwise
    \end{cases*}
    \forall y \in \varSet
\end{equation*}

[CONJECTURE AND ASSUMPTIONS]