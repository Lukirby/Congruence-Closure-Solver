\chapter{Propositional Logic}
\label{cha:Propositional Logic}

Propositional logic is a formal system that deals with propositions,
which are statements that are either true or false.

\section{Syntax of Propositional Logic}
\label{sec:Syntax of Propositional Logic}

It is composed of a set $\propSet$ of symbols called \textit{propositional variables}
which are denoted by $P,Q,R,\dots$ , $x,y,z,\dots$ , $1,2,3,\dots$ or $true , false$.

An atomic variable are:
\begin{itemize}
    \item $\top$ which is always true.
    \item $\bot$ which is always false.
    \item $P$ which is a propositional variable.
\end{itemize}

The logical connectives are $\neg$ (negation), $\land$ (conjunction), $\lor$ (disjunction),
$\implies$ (implication) and $\iff$ (equivalence).

\begin{definition}{Propositional Formula}
    A \textbf{propositional formula} is defined as follows:
    \begin{itemize}
        \item Every atomic variable is a formula.
        \item If $F$ and $G$ are formulas, then $\neg F$, $F \land G$, $F \lor G$, 
        $F \implies G$, $F \iff G$ are formulas.
    \end{itemize}    
\end{definition}

\begin{definition}{Propositional Interpretation}
    A \textbf{propositional interpretation} of a formula is a function that assigns 
    a truth value to each atomic variable, and it is defined as follows: 
    $\I : \propSet \to \{true,false\}$.    
\end{definition}

\section{Semantic in Propositional Logic}
\label{sec:Satisfiability in Propositional Logic}

An interpretation $\I$ is said to \textit{satisfy} a formula $F$, 
written as $\I \vDash F$:
\begin{itemize}
    \item if $F$ is an atomic variable $P$, then $\I(P) = true$.
    \item if $F = \neg G$, then $\I \vDash F$ if $\I \nvDash G$.
    \item if $F = G \land H$, then $\I \vDash F$ if $\I \vDash G$ and $\I \vDash H$.
    \item if $F = G \lor H$, then $\I \vDash F$ if $\I \vDash G$ or $\I \vDash H$.
    \item if $F = G \implies H$, then $\I \vDash F$ if $\I \nvDash G$ or $\I \vDash H$.
    \item if $F = G \iff H$, then $\I \vDash F$ if $\I \vDash G$ and $\I \vDash H$ or
    $\I \nvDash G$ and $\I \nvDash H$.
\end{itemize}

\begin{definition}{Satisfiable Formula}
    A formula $F$ is \textbf{satisfiable} if $\exists \I$ 
    interpretation such that $\I \vDash F$.
\end{definition}

\begin{definition}{Valid Formula}
    A formula $F$ is \textbf{valid} if $\forall \I$ 
    interpretation such that $\I \vDash F$.
\end{definition}

\begin{remark}{Unsatisfiable and Invalid Formulas}
    A formula $F$ is \textbf{unsatisfiable} if $\forall \I$ interpretation 
    such that $\I \nvDash F$.
    A formula $F$ is \textbf{invalid} if $\exists \I$ interpretation 
    such that $\I \nvDash F$.
\end{remark}

\begin{remark}{Implication of Validity}
    Let $F$ be a formula, than we can observe that:
    \maketable{
        \hline
        $F$ & & $\neg F$ \\
        \hline
        Satisfiable & $\implies$ & Invalid \\
        Valid & $\implies$ & Unsatisfiable \\
        Invalid & $\implies$ & Satisfiable \\
        Unsatisfiable & $\implies$ & Valid \\
        \hline
    }{}{|c|c|c|}
\end{remark}

[IDEA OF THE PROBLEM OF SATISFIABILITY IN PROPOSITIONAL LOGIC]
[NOTE IN INTRODUZIONE 2 OTTOBRE]

Given a formula $F$, than $F$ is finite, hence the number of propositional
variables is finite, therefore the number of interpretations is finite.

In particular the number of interpretations is $2^{n}$, where $n$
is the number of propositional variables.

[SCHEMA AD ALBERO] [INTRODUZIONE 2 OTTOBRE]

testing every interpretation is costly and inefficient,
we need to design a decision procedure that is able to determine the satisfiability
of a formula in a more efficient way, using normal forms.

[DECISION PROCEDURE] [INTRODUZIONE 2 OTTOBRE]

\section{Normal Forms}
\label{subsec:Normal Forms}

\subsection{Negation Normal Form}
\label{subsec:Negation Normal Form}

\begin{definition}{Negation Normal Form}
    A formula $F$ is in \textbf{negation normal form} if
    the only connective that appears is $\neg, \land, \lor$ and 
    $\neg$ is only applied to atomic variables. 
\end{definition}

The procedure to convert a formula $F$ into negation normal form is as follows:
\begin{itemize}
    \item $\neg \neg G \equiv G$.
    \item $\neg (G \land H) \equiv \neg G \lor \neg H$.
    \item $\neg (G \lor H) \equiv \neg G \land \neg H$.
    \item $(G \implies H) \equiv \neg G \lor H$.
    \item $(G \iff H) \equiv (\neg G \lor H) \land (G \lor \neg H)$
\end{itemize}

\subsection{Disjunctive Normal Form}
\label{subsec:Disjunctive Normal Form}

\begin{definition}{Disjunctive Normal Form}
    A formula $F$ is in \textbf{disjunctive normal form} if
    it is a disjunction of conjunctions of atomic variables.

    \begin{equation}
        F = D_{1} \lor D_{2} \lor \dots \lor D_{n}
    \end{equation}

    where $D_{i} = (L_{1}^i \land L_{2}^i \land \dots \land L_{n}^i)$ is a 
    conjunction of atomic variables called \textit{cube}.
\end{definition}

The procedure to convert a formula $F$ into disjunctive normal form is
to convert it into negation normal form and then apply the distributive law:
\begin{itemize}
    \item $G \land (H \lor K) = (G \land H) \lor (G \land K)$.
    \item $(G \lor H) \land K = (G \land K) \lor (H \land K)$.
\end{itemize}

\subsection{Conjunctive Normal Form}
\label{subsec:Conjunctive Normal Form}

\begin{definition}{Conjunctive Normal Form}
    A formula $F$ is in \textbf{conjunctive normal form} if
    it is a conjunction of disjunctions of atomic variables.

    \begin{equation}
        F = C_{1} \land C_{2} \land \dots \land C_{n}
    \end{equation}

    where $C_{i} = (L_1^i \lor \dots \lor L_n^i)$ is a disjunction of atomic variables
    called \textit{clause}.
\end{definition}

The procedure to convert a formula $F$ into conjunctive normal form is 
to convert it into negation normal form and then apply the distributive law:
\begin{itemize}
    \item $G \lor (H \land K) = (G \lor H) \land (G \lor K)$.
    \item $(G \land H) \lor K = (G \lor K) \land (H \lor K)$.
\end{itemize}

For the SAT problem, the conjunctive normal form is the most used normal form,
because it is the most efficient to determine the satisfiability of a formula.

The distributive law cause the formula to grow exponentially,
hence adding cost to the decision procedure.