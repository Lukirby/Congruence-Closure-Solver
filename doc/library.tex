
\graphicspath{ {./images/} }

%\makefigure{figure.ext}{caption}{width}
\newcommand{\makefigure}[3]{
    %\begin{figure}[position] => {h=here, t=top, b=bottom, H=here + ignore limitation}
    \begin{figure}[H]
        \begin{center}
            \includegraphics[width=#3]{#1} 
        \end{center}
        \caption{#2}
        \label{fig:#1}
    \end{figure}
}

%\makesubfigure{figure.ext}{caption}{width}{widthFigure}
\newcommand{\makesubfigure}[4]{
    %\begin{subfigure}[position]{width}
    \begin{subfigure}[H]{#3}
        \centering
        \includegraphics[width=#3]{#1}
        \caption{#2}
        \label{fig:#1}
    \end{subfigure}
}

%\maketable{content}{caption}{width}
\newcommand{\maketable}[3]{
    \begin{table}[H]
        \begin{center}  
            \begin{tabular}[H]{#3}
                #1
            \end{tabular}
        \end{center}
        \ifstrempty{#2}{}
        {
            \caption{#2}
            \label{tab:#2}
        }
    \end{table}
}

% \makegraphic{opzioni assi}{plots e disegni}
\newcommand{\makegraphic}[2]{
    \begin{center}
    \begin{tikzpicture}
        \begin{axis}[#1]
            #2
        \end{axis}
    \end{tikzpicture}
    \end{center}
}

% \makesection{titolo}{aggiunta_label}
\newcommand{\makesection}[2]{
    \section{#1}
    \label{sec#2:#1}
}

% \makesection{titolo}{aggiunta_label}
\newcommand{\makesubsection}[2]{
    \subsection{#1}
    \label{sec#2:#1}
}

\definecolor{codegreen}{rgb}{0,0.6,0}
\definecolor{codegray}{rgb}{0.5,0.5,0.5}
\definecolor{codepurple}{rgb}{0.58,0,0.82}
\definecolor{backcolour}{rgb}{0.95,0.95,0.92}

\lstdefinestyle{mystyle}{
    backgroundcolor=\color{white},   
    commentstyle=\color{codegreen},
    keywordstyle=\color{magenta},
    numberstyle=\tiny\color{codegray},
    stringstyle=\color{codepurple},
    basicstyle=\ttfamily\footnotesize,
    breakatwhitespace=false,         
    breaklines=true,                 
    captionpos=b,                    
    keepspaces=true,                 
    numbers=left,                    
    numbersep=5pt,                  
    showspaces=false,                
    showstringspaces=false,
    showtabs=false,                  
    tabsize=2,
    linewidth=0.8\linewidth,
    inputpath={./codes/},
    float=c
}

\lstset{style=mystyle}

%\code{code.ext}{capiton}{language}
\newcommand{\code}[3]{
    \begin{figure}[H]
    \centering \begin{tabular}{c}
        \lstinputlisting[language=#3]{#1}
    \end{tabular}
    \ifstrempty{#2}{}
    {
        \caption{#2}
        \label{tab:#1}
    }
    \end{figure}
}