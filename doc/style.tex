\usepackage[most]{tcolorbox}

\colorlet{coldef}{red!75!black}
\colorlet{colexa}{blue!65!white}
\colorlet{colrem}{green!65!black}
\colorlet{colthe}{magenta!85!black}
\colorlet{colnot}{orange!85!black}
\colorlet{colgra}{blue!25!black}

% Definizione del contatore
\newcounter{boxcounter}[section]

% Impostazione della numerazione progressiva
\numberwithin{boxcounter}{section}

% \newtcbtheorem[init options]{display name}{title}{options}{prefix}
\newtcbtheorem[use counter = boxcounter,number within=section]{Definition}{Definition}{
  lower separated=false,
  colback = white,
  colframe = coldef, 
  fonttitle=\bfseries,
  colbacktitle = coldef,
  coltitle=white,
  enhanced,
  %attach boxed title to top left={yshift=0in,xshift=0in},
  %boxed title style={boxrule=0pt,colframe=white,},
  }{def}

\newenvironment{definition}[1]{
  \begin{Definition}{#1}{#1}
}{
  \end{Definition}
}

\newtcbtheorem[use counter = boxcounter,number within=section]{Example}{Example}{
  lower separated=false,
  colback = white,
  colframe = colexa, 
  fonttitle=\bfseries,
  colbacktitle = colexa,
  coltitle=white,
  enhanced,
  %attach boxed title to top left={yshift=0in,xshift=0in},
  %boxed title style={boxrule=0pt,colframe=white,},
  }{exa}

\newenvironment{example}[1]{
  \begin{Example}{#1}{#1}
}{
  \end{Example}
}

\newtcbtheorem[use counter = boxcounter,number within=section]{Notation}{Notation}{
  lower separated=false,
  colback = white,
  colframe = colnot, 
  fonttitle=\bfseries,
  colbacktitle = colnot,
  coltitle=white,
  enhanced,
  %attach boxed title to top left={yshift=0in,xshift=0in},
  %boxed title style={boxrule=0pt,colframe=white,},
  }{not}
 
\newenvironment{notation}[1]{
  \begin{Notation}{#1}{#1}
}{
  \end{Notation}
}

\newtcbtheorem[use counter = boxcounter,number within=section]{Remark}{Remark}{
  lower separated=false,
  colback = white,
  colframe = colrem, 
  fonttitle=\bfseries,
  colbacktitle = colrem,
  coltitle=white,
  enhanced,
  %attach boxed title to top left={yshift=0in,xshift=0in},
  %boxed title style={boxrule=0pt,colframe=white,},
  }{rem}

\newenvironment{remark}[1]{
  \begin{Remark}{#1}{#1}
}{
  \end{Remark}
}

\newtcbtheorem[use counter = boxcounter,number within=section]{Theorem}{Theorem}{
  lower separated=false,
  colback = white,
  colframe = colthe, 
  fonttitle=\bfseries,
  colbacktitle = colthe,
  coltitle=white,
  enhanced,
  %attach boxed title to top left={yshift=0in,xshift=0in},
  %boxed title style={boxrule=0pt,colframe=white,},
  }{the}

\newenvironment{theorem}[1]{
  \begin{Theorem}{#1}{#1}
}{
  \end{Theorem}
}

\newtcbtheorem[no counter,number within=section]{Proof}{Proof\!\!}{
  lower separated=false,
  colback = white,
  colframe = colthe, 
  fonttitle=\bfseries,
  colbacktitle = colthe,
  coltitle=white,
  enhanced,
  separator sign={}
  %attach boxed title to top left={yshift=0in,xshift=0in},
  %boxed title style={boxrule=0pt,colframe=white,},
  }{}

\newenvironment{prooff}[1]{
  \begin{Proof}{\ref{the:#1}: #1}{}
}{
  \hfill$\qed$
  \end{Proof}
}

%continued proof environment
\newenvironment{cprooff}[1]{
  \begin{Proof}{\ref{the:#1}: #1}{}
}{
  \hfill$\longrightarrow$
  \end{Proof}
}

\newtcbtheorem[use counter = boxcounter,number within=section]{Graphic}{Graphic}{
  lower separated=false,
  colback = white,
  colframe = colgra, 
  fonttitle=\bfseries,
  colbacktitle = colgra,
  coltitle=white,
  enhanced,
  %attach boxed title to top left={yshift=0in,xshift=0in},
  %boxed title style={boxrule=0pt,colframe=white,},
  }{gra}

\newenvironment{graphic}[1]{
  \begin{Graphic}{#1}{#1}
}{
  \end{Graphic}
}

\newtcbtheorem[use counter = boxcounter,number within=section]{Table}{Table}{
  lower separated=false,
  colback = white,
  colframe = colgra, 
  fonttitle=\bfseries,
  colbacktitle = colgra,
  coltitle=white,
  enhanced,
  %attach boxed title to top left={yshift=0in,xshift=0in},
  %boxed title style={boxrule=0pt,colframe=white,},
  }{tab}

\newenvironment{tablee}[1]{
  \begin{Table}{#1}{#1}
}{
  \end{Table}
}

\newtheoremstyle{defstyle}% Nome dello stile
  {\topsep}% Spazio sopra
  {\topsep}% Spazio sotto
  {}% Font del corpo del teorema
  {}% Indentazione del corpo del teorema
  {\bfseries}% Font dell'intestazione
  {}% Punteggiatura dopo l'intestazione
  {\newline}% Spazio dopo l'intestazione (\newline equivale a un a capo)
  {\thmname{#1}\thmnumber{ #2} \thmnote{(#3)}:}% Formato dell'intestazione


% cambio il simbolo di itemize dentro itemize
\renewcommand{\labelitemii}{\textopenbullet}
